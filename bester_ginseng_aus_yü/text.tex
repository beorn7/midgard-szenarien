\documentclass[
a4paper,
twoside,
DIV=calc,
BCOR=4mm,
fontsize=9pt,
twocolumn=on,
titlepage=on
]{scrartcl}
\usepackage{german,multicol,supertabular,graphicx,hyperref}
\usepackage[utf8]{inputenc}
%\usepackage[T1]{fontenc}
%\usepackage{helvet}
%\usepackage{mathphv}

%\renewcommand{\familydefault}{phv}
%\renewcommand{\familydefault}{cmss}
%\setlength{\columnsep}{7mm}

%\usepackage{anysize}
%          {left}{right}{top}{bottom}
%\marginsize{2.0cm}{2.0cm}{2.0cm}{2.0cm}
%\setlength{\footnotesep}{5mm}

\hyphenation{NSpF}
\hyphenation{Re-gel-mo-di-fi-ka-tio-nen}

\newcommand{\midgard}{\textsc{Midgard}}

\author{Bj"orn~Rabenstein\\\href{mailto:bjoern@rabenste.in}{\nolinkurl{bjoern@rabenste.in}}}
\title{Bester Ginseng aus Yü}
\subtitle{\midgard-Abenteuer für 3--5 SpF der Grade 1--3}

\publishers{\fbox{%
\parbox{\columnwidth}{%
Dieses Material steht unter der Creative-Commons-Lizenz \emph{Namensnennung
-- Nicht-kommerziell -- Weitergabe unter gleichen Bedingungen 4.0
International}. Um eine Kopie dieser Lizenz zu sehen, besuchen Sie
\url{http://creativecommons.org/licenses/by-nc-sa/4.0/}.
}%
}}


\begin{document}

\maketitle


\tableofcontents

%\begin{multicols}{2}

\section{Einleitung}

Im Januar 2015 erfüllte ich mir so eine Art Jugendtraum und begann,
eine KanThaiPan-Kampagne zu leiten, in deren Zentrum die "`vier
Klassiker"', also die vier Richter-Di-Abenteuer, stehen sollten. Ein
Vorteil der späten Traumerfüllung war, daß wir wirklich mit dem
\emph{Fall des Mondloses} anfangen konnten. Dadurch stellte sich aber
die Frage, wie wir die gut zehnjährige Lücke zwischen dem
\emph{Mondlos} und den \emph{Füchsen} überbrücken sollten. Selbst mit
einem bißchen Füllmaterial und Tricks blieb bei uns eine Lücke von
2392\,nL bis 2400\,nL. Schickte man die SpF wirklich so lange auf
Abenteuer aus, dann wären sie danach so ca. auf Gr9, was dann doch
etwas heftig wäre für den Rest der Kampagne, ganz abgesehen mal davon,
wie lange man dafür bräuchte. Natürlich hätte ich jetzt die SpF acht
Jahre lang ihrem "`normalen"' Leben nachgehen lassen können, oder ich
hätte die Geschichte umschreiben können und die anderen
Richter-Di-Abenteuer früher stattfinden lassen. Beide Möglichkeiten
fand ich aber aus verschiedenen Gründen unbefriedigend. So habe ich
aus der Not eine Tugend gemacht und dieses Abenteuer entworfen, wo als
Teil der Handlung die SpF einen Zeitsprung unternehmen werden, und das
sogar ohne die naheliegende \emph{Mach über die Zeit}, sondern unter
Ausnutzung bekannter Eigenschaften des Multiversums.

Um mit dieser Klappe gleich noch ein paar andere Fliegen zu schlagen,
passieren noch ein paar andere Dinge:

\begin{itemize}
\item Die SpF lernen Ming kennen, den wichtigen Protagonisten aus
  \emph{Kurai-Anat, das Schwarze Herz}. Das ermöglicht es, später den
  Kernteil dieses Abenteuers auch mit einer reinen KanThai-Gruppe zu
  spielen. Ein paar Szenen aus dem Abenteuer, die sonst gar nicht mehr
  zur Geltung kämen, können hier auch gleich verwurstet werden.
\item In den \emph{Perlen der Füchse} schmuggelt der \emph{Weiße
    Lotus} Gingseng in weitem Bogen von irgendwo am Schattenmeer durch
  Minangpahit über den Seeweg nach KuenKung. Dabei liegt doch der
  Hauptproduzent von Gingseng, die Provinz Yü, mehr oder weniger
  direkt vor den Toren der Stadt. Dieses Abenteuer liefert eine
  Erklärung, wie es dazu kommen konnte. Obendrein erfährt man ein
  wenig über das Verhältnis zwischen \emph{Weißer Orchidee} und
  \emph{Weißem Lotus}. Letzterer war ja zu Beginn des Abenteuers noch
  keine eigenständige Geheimgesellschaft, sondern lediglich eine Loge
  von ersterer.
\end{itemize}

Ich kann mir vorstellen, daß auch andere Gruppen in der gleichen
Situation stecken. Daher habe ich mich entschlossen, das Abenteuer
hiermit der Allgemeinheit zur Verfüngung zu stellen. Ich bitte, den
etwas rohen Ausarbeitungszustand zu verzeihen. Ich habe im
wesentlichen auf die Schnelle meine Notizen ausformuliert. Karten und
Skizzen sind in dem beklagenswerten Zustand eingescannt worden, in dem
sie mein bescheidenes manuelles Zeichentalent hinterlassen hat.

Verweise auf Regelbücher, das KanThaiPan-Quellenbuch und diverse
Abenteuerhefte erfolgen nach dem Schema [\emph{XXX n}], wobei
\emph{XXX} die Abürzung für das Buch bzw. Heft ist (siehe folgende
Liste) und \emph{n} die Seitenzahl im jeweiligen Werk.

\begin{description}
\item[DFR] \emph{Das Fantasy-Rollenspiel} -- \midgard-Regelwerk 4. Auflage.
\item[DSH] \emph{Kurai-Anat, das Schwarze Herz} -- Abenteuer.
\item[KTP] \emph{Unter dem Schirm des Jadekaisers} -- Quellenbuch
  KanThaiPan 2. Auflage.
\item[MdS] \emph{Meister der Sphären} -- \midgard-Regelwerk 4. Auflage.
\item[PdF] \emph{Die Perlen der Füchse} -- Abenteuer.
\end{description}

Dieses Abenteuer beziehte sich auf die 4. Auflage des \midgard-Regelwerkes.

\section{Überblick}

Das KuraiAnat sorgt sich um die Versorgung mit Ginseng. Daß
vergleichsweise unabhängige Händler aus den Küstenregionen, denen
sogar die Zusammenarbeit mit feindlichen Geheimgesellschaften wie der
\emph{Weißen Orchidee} zuzutrauen ist, wesentliche Teile des
Ginsenghandels unter ihrer Kontrolle bringen, ist eine nicht zu
unterschätzende Gefahr. Der schwarze Adept TsueChen wird beauftragt,
die Sache nachhaltig in Ordnung zu bringen. Dafür verbündet er sich
mit der dem KuraiAnat loyal ergebenen Familie Xuang aus KuenKung, die
bereits einen guten Teil des Ginsenghandels mit dem Volk der Yü, das
den meisten Ginseng in KanThaiPan produziert, kontrolliert. Zusammen
mit den Xuang übt er Druck auf die Yü aus, ihre Ginsengernte nur noch
an die Familie Xuang zu verkaufen. Mittelfristig soll dieses Monopol
durch einen kaiserlichen Erlaß Gesetzeskraft erlangen.

Die Yü sind nicht zentral organisiert, aber sie haben eine Älteste,
deren Rat von den Yü nicht leichtfertig in den Wind geschlagen
wird. Die aktuelle Amtsinhaberin ist AMei. Auf sie üben TsueChen und
LiXuang, der zuständige Händler der Familie Xuang, besonders viel
Druck aus. Ihr Plan scheint aufzugehen, den AMei zeigt sich einsichtig
und rät ihren Volksgenossen, den "`Vorschlag"' der Xuang anzunehmen.

Die \emph{Weiße Orchidee}, die sich in der Tat im Ginsenghandel
engagiert, ist davon gar nicht begeistert. Sie versuchen einerseits,
Ginsengsammler der Yü auf ihre Seite zu bringen. Andererseits
sabotieren sie den Ginsenghandel der Xuang durch Überfälle auf ihre
Händler. Die \emph{Weiße Orchidee} wird insgeheim von der
Fürstenfamilie von KuenKung, den Tchung, unterstützt, ebenso von ihrer
vorwiegend durch Beamte und etablierte Personen des öffentlichen
Lebens gebildeten Loge des \emph{Weißen Lotus}. Insbesondere beim
\emph{Weißen Lotus} regen sich aber immer mehr Zweifel am Erfolg
dieser Strategie.

DiFang ist ein besonders erfolgreicher Ginsengsammler, der
traditionell mit der Familie Ming aus KuenKung zusammenarbeitet. Kurz
vor Beginn der Handlung des Abenteuers hat er sich aber entschlossen,
dem Drängen AMeis nachzugeben und seinen Ginseng nur noch an die Xuang
zu liefern. Er schreibt eine Nachricht nieder, die die Hintergründe
seiner Entscheidung sehr klar beschreibt, und gibt sie HaiFeng, dem
Kapitän im Diensten der Familie Ming, mit. Die Xuang bekommen davon
Wind. Weil sie nicht wollen, daß Beweise für ihre Aktivitäten an das
Licht der Öffentlichkeit geraten, überfallen sie HaiFeng und nehmen
ihm die versiegelte Botschaft ab.

An dieser Stelle betreten die SpF die Bühne des Geschehens. Sie finden
den bewußtlosen HaiFeng und werden früher oder später vom Händler
MingMing, dem Oberhaupt der Familie Ming, beauftragt, den Verbleib von
DiFang aufzuklären.

Die Ermittlungen führen die SpF in die Provinz Yü, und zwar
wahrscheinlich zunächst in das Fischerdorf OnchiRa, wo sie die Chance
erhalten, Zeuge des Wirkens eines YamaOni zu werden, der die Frau des
Dorfvorstehers zu seiner Geliebten erkoren hat.

DiFang hat ein Geheimnis: Er findet die großen Mengen besonders
qualitativen Ginsengs nicht im Land der Yü, ja noch nicht mal auf
Midgard. Er hat Kenntnis von einem uralten permanenten Weltentor nach
RenSchenYo, einer Urwelt, von der der Ginseng ursprünglich stammt. Den
Bedrängungen sowohl durch die Xuang als auch durch die \emph{Weiße
  Orchidee} entzieht er sich durch Flucht nach RenSchenYo. Dort wird
er Opfer eines Streiches des (weiblichen) Dschinns BaiSseYuen, ihres
Zeichens Lehrmeisterin von Ming, der insgeheim ein Elementarbeschwörer
ist. Er sitzt vorerst auf RenSchenYo fest. Da die Zeit in einer Urwelt
schneller vergeht als auf Midgard, haben die SpF auf Midgard viel
Zeit, die Spur DiFangs aufzunehmen oder auf andere Art und Weise auf
das Weltentor aufmerksam zu werden und schließlich DiFang aus seiner
mißlichen Lage zu retten.

Bis es dazu kommt, können die SpF noch so einiges über die politischen
Ereignisse rund um den Ginsenghandel herausfinden. Auch könnten sie
KokoRo über den Weg laufen, der vermißten Schwester einer der SpF, die
mittlerweile zu einer Orchideenklinge ausgebildet worden ist.

Während die SpF DiFang aus seiner Lage auf RenSchenYo retten, vergehen
auf Midgard knapp acht Jahre. Bei ihrer Rückkehr nach Midgard finden
sie einen komplett durch die Xuang monopolisierten Ginsenghandel durch
die Yü vor. Ming hat sich mittlerweile anderen Handelsaktivitäten,
insbesondere dem Fernhandel mit dem Westen Midgards,
zugewandt. Dennoch ist er hocherfreut über die Errettung seines
Freundes DiFang und sehr begierig, von den Abenteuern der SpF zu
erfahren. Der \emph{Weiße Lotus} hat sich weitgehend von der
\emph{Weißen Orchidee} gelöst und ähnelt nun eher einer eigenen
Geheimgesellschaft. Er hat ausländische Quellen für Ginseng
erschlossen, während die \emph{Weiße Orchidee} nur noch sehr
halbherzig versucht, direkt gegen den Ginsenghandel der Xuang
vorzugehen.

In vielen der folgenden Abschnitte finden sich Absätze, die mit
\textsc{Zukunft:} beginnen. Diese beschreiben Entwicklungen und
Ereignisse, die erst stattfinden, wenn die SpF sich auf RenSchenYo
befinden und etliche Jahre auf Midgard überspringen. Das Abenteuer
geht davon aus, daß die SpF im Winter 2400 wieder auf Midgard
eintreffen. Der SpL kann dies an die Bedürfnisse seiner Kampagne
anpassen, muß dann aber ggf. Entwicklungen und Ereignisse entsprechend
abändern.

\section{Dramatis personae}

\begin{description}
\item[MingMing.] Händler der Weiße-Schlangen-Gilde in KuenKung und
  heimlicher Elementarbeschwörer.
\item[DiFang.] Ginsengsammler aus dem Volk der Yü. Geschäftspartner
  und Vertrauter Mings.
\item[HaiFeng.] Kapitän in Diensten Mings. Opfer der KuroScha.
\item[ChenXuang.] Wenig integrer Richter von KuenKung, Vorgänger von
  DiYung.
\item[LiXuang.] Händler aus KuenKung. Spezialist für den
  Ginsenghandel mit den Yü.
\item[AMei.] Älteste der Yü. Ansprechpartnerin von LiXuan.
\item[TsueChen.] Schwarzer Adept mit stetig sich verbesserndem
  Verhältnis zur Familie Xuang. Zuständig für die Ginsengversorgung
  des KuraiAnat.
\item[MuLanLu.] Renommierteste YiScheng (Ärztin) von
  KuenKung. Mitglied des \emph{Weißen Lotus}.
\item[KwanLi.] Misogyner Dorfältester von OnchiRa.
\item[MeiHua.] Seine Frau.
\item[Yao.] Ihr heimlicher Liebhaber. In Wirklichkeit ein YamaOtoko.
\item[KokoRo.] Orchideenklinge, Schwester einer SpF, vermeintlich
  den \emph{Weg der Tausend} gegangen.
\item[BaiSseYuen -- "`Weiße Wolke"'.] Dschinn. Lehrmeisterin von Ming.
\item[MiTo.] Priester des KuTuh, z.Zt. Abt im Kloster DaKuMi. Verwandt
  mit den Xuang in KuenKung, wo er im Laufe der 90er Jahre zum
  Hohepriester des örtlichen Tempel der Düsternis aufsteigen wird.
\end{description}

Agenten der \emph{Weißen Orchidee}, Mitglieder der Familie Xuan und
deren Schergen, KuroScha, Ginsengsammler der Yü, Meerjungfrauen, Wako,
Hunar, Luftrochen\dots

\section{Zeitablauf}

Die Datumsangaben erfolgen nach westlichem Muster (mit dem Bärenmond
als ersten Monat des Jahres). Zur Umrechnung in kanthanische
Jahresangaben siehe folgende Liste.

\begin{itemize}
\item 01. Wolf 2391\,nL -- 28. Draug 2392\,nL: GuiYou (Hahn)
\item 01. Wolf 2392\,nL -- 28. Draug 2393\,nL: JiaXu (Hund)
\item 01. Wolf 2393\,nL -- 28. Draug 2394\,nL: YiHai (Eber/Schwein)
\item 01. Wolf 2394\,nL -- 28. Draug 2395\,nL: BingZi (Ratte)
\item 01. Wolf 2395\,nL -- 28. Draug 2396\,nL: DingChou (Ochse)
\item 01. Wolf 2396\,nL -- 28. Draug 2397\,nL: WuYin (Tiger)
\item 01. Wolf 2397\,nL -- 28. Draug 2398\,nL: JiMao (Hase)
\item 01. Wolf 2398\,nL -- 28. Draug 2399\,nL: GengLong (Drache)
\item 01. Wolf 2399\,nL -- 28. Draug 2400\,nL: XinSi (Schlange)
\item 01. Wolf 2400\,nL -- 28. Draug 2401\,nL: RenWu (Pferd)
\end{itemize}

\subsection{Vor Beginn des Abenteuers}

\begin{description}
\item[In grauer Vorzeit:] Die Arracht identifizieren die
  Neun-Blumen-Berge als besonders geeignet, ein Tor in eine Urwelt der
  Luft/Holz-Sphäre zu errichten. Im Süden der heutigen Provinz Yü
  errichten sie schließlich ein Weltentor nach RenSchenYo. Die
  wichtigste Entdeckung auf dieser Welt war der Ginseng, der von den
  Arracht auch auf Midgard angepflanzt wurde, allerdings mit wesentlich
  geringerem Erfolg.
\item[um 1000\,vL:] Mit der Entwicklung menschlicher Zivilisation in
  KanThaiPan entsteht auch das Volk der Yü samt ihrer Religion des
  Ginsenggottes. Die Priester belegen das Weltentor mit einem Tabu und
  machen es für Menschen unauffindbar.
\item[2337\,nL:] Der junge Ziegenhirte DiFang entdeckt beim Verfolgen
  einer Ziege das Weltentor. Er folgt seiner Ziege nach RenSchenYo,
  kann aber das Tor nicht umpolen, um zurückzukehren.
\item[2343\,nL:] Durch einen Glücksfall kann DiFang einen Baumriesen
  beim Umpolen des Tores beobachten und so nach Midgard zurückkehren,
  wo in den wenigen Stunden, die er auf RenSchenYo verbracht hat,
  sechs Jahre vergangen sind. Fortan nutzt DiFang die Kenntnis über
  Ort und Umpolung des Weltentores, um zu einem legendären
  Ginsengsammler aufzusteigen. Zum Handelspartner seines Vertrauens
  entwickelt sich die Familie Ming aus KuenKung.
\item[2385\,nL:] Der junge MingMing widmet sich dem Studium des
  Multiversums. DiFang weiht ihn in sein Geheimnis ein und ermöglicht
  ihm einen Besuch RenSchenYos, wo er schließlich auf seine spätere
  übernatürliche Lehrerin BaiSseYuen trifft.
\item[2391\,nL:] Der Schwarze Adept TsueChen wird vom Kronrat zum
  Beauftragten für den Ginsenghandel ernannt. Seine Aufgabe ist es,
  die Ginsengversorgung des KuraiAnat sicherzustellen. Er nimmt mit
  der Familie Xuan auf, die bereits einen Großteil des Ginsenghandels
  mit den Yü kontrolliert.
\item[Herbst 2392\,nL:] Mit TsueChens Unterstützung übt LiXuang
  zunehmend Druck auf die Yü aus, nur noch mit der Familie Xuang zu
  handeln. Die \emph{Weiße Orchidee} beginnt, den Ginsenghandel der
  Familie Xuang aktiv zu sabotieren und Ginseng in eigene Kanäle
  umzulenken.
\item[1. Draug 2392\,nL:] DiFang, der sich mittlerweile außerstande
  sieht, gegen den Druck der anderen Yü weiterhin Ginseng an Ming zu
  liefern, trift in OnchiRa ein, um dem dort wartenden HaiFeng statt
  der erwarteten Ginsenglieferung ein versiegeltes Schreiben zu
  überreichen. Kurz nach Abfahrt HaiFengs treffen drei Xuang-SaMurai
  ein, die die Aufgabe haben, DiFangs Geheimnis zu enträtseln. Sie
  bekommen mit, daß DiFang HaiFeng einen Brief übergeben hat. Diese
  Information übermitteln sie per Brieftaube nach KuenKung.
\item[2. Draug 2392\,nL:] DiFang verläßt OnchiRa, heimlich verfolgt
  von den drei Xuang-SaMurai.
\item[3. Draug 2392\,nL:] DiFang trifft im Winterlager AMeis ein und
  verhandelt mit ihr über die zukünftigen Konditionen.
\item[4. Draug 2392\,nL:] DiFang reist weiter in Richtung Schattental,
  weiterhin von den SaMurai verfolgt.  Eine Patrouille von vier
  Orchideenklingen (unter ihnen KokoRo), die ebenfalls auf der Suche
  nach DiFang sind und seine Spur bei AMeis Lager aufnehmen wollten,
  heftet sich unauffällig an die Fersen der SaMurai. -- Gegen Abend
  kommt HaiFeng in KuenKung an und will sofort den Brief an Ming
  übergeben. Noch im Hafenbereich betäubt ihn aber hinterrücks ein
  KuroScha im Auftrag der Xuans und entwendet den Brief. Die SpF
  finden den betäubten HaiFeng.
\end{description}

\subsection{Nach Beginn des Abenteuers}

Dies ist der wahrscheinliche Ablauf. Je nach Aktionen der SpF kann
sich natürlich einiges ändern.

\begin{description}
\item[5. Draug 2392\,nL:] Ming zeigt den Überfall auf HaiFeng bei
  Gericht an, wobei ihm klar wird, daß von Seiten der Behörden keine
  Bemühungen um Aufklärung zu erwarten sind. Spätestens jetzt
  beauftragt er die SpF, dem Fall nachzugehen. -- Auf RenSchenYo wird
  DiFang von BaiSseYuen auf die nächste Insel entführt. Da das Tor
  immer noch umgepolt ist, dringen ab jetzt gelegentlich Luftrochen
  nach Midgard vor.
\item[6. Draug 2392\,nL:] Am Eingang des Schattentals bemerkt DiFang
  seine Verfolger, die ihn daraufhin gefangennehmen wollen. Dies ist
  für die Orchideenklingen Anlaß, nun einzugreifen. Sie töten zwei der
  SaMurai, der dritte entkommt Richtung KuTuh-Kloster. Die
  Orchideenklingen bedrängen dann DiFang, sich ihrer Seite
  anzuschließen. DiFang gerät in Panik und flieht durchs Weltentor,
  was er daraufhin sofort umpolt, um seine Verfolger
  abzuschneiden. Die Orchideenklingen schlagen ein verborgenes Lager
  auf, um die weitere Entwicklung zu beobachten. DiFangs Kummer lockt
  BaiSseYuen an, die sich gerade auf RenSchenYo aufhält und ihn
  durch einen Streich vorläufig dort festsetzt.
\item[7. Draug 2392\,nL:] Frühestmögliches Eintreffen der SpF in
  OnchiRa. Mögliche Beobachtung des Zusammentreffens von Yao und
  MeiHua. (Auch an späteren Tagen möglich.)
\item[12. Draug 2392\,nL:] Frühestmögliches Eintreffen der SpF am
  Weltentor. Mögliches Eintreffen der Verstärkung der Xuang-Krieger am
  Weltentor.
\item[Drachenmond 2393\,nL:] MeiHua bringt in OnchiRa ihren Sohn
  Chang zur Welt.
\item[2394\,nL:] Es ergeht eine kaiserliche Verordnung zur
  Monopolisierung des Ginsenghandels mit den Yü. Die dahingehende
  Lizenz wird der Familie Xuan erteilt. Die Yü beugen sich, und der
  \emph{Weiße Lotus} erschließt andere Quellen für den Ginsenghandel,
  während die \emph{Weiße Orchidee} noch längere Zeit versucht, das
  Monopol durch Überfälle und Verbündete bei den Yü zu unterlaufen.
\item[3497\,nL:] Richter ChenXuan wir pensioniert. DiYung wird (nicht
  ganz ohne Einflußnahme seiner Familie) zum neuen Richter von
  KuenKung ernannt.
\item[Trollmond 2400\,nL:] Etwa zu diesem Zeitpunkt kehren die SpF
  und DiFang nach Midgard zurück. -- In OnchiRa zeigt Chang immer mehr
  bestialische Züge. Es werd Zeit für Yao, seinen Sohn an sich zu
  nehmen\dots
\end{description}

\section{Ermittlungen in KuenKung}

\subsection{Handlungsunfähig im Handelshafen}

Am Abend des 4. Draug 2392\,nL (also am 4. Tag des 13. Monats im Jahr
GuiYou) finden die SpF im Handelshafenviertel einen gut gekleideten
Seemann bewußtlos in einer dunklen Ecke liegend (z.\,B. auf dem
Heimweg, wenn sie in der \emph{Meeresbrise} [PdF\,15f] abgestiegen
sind).

Der Bewußtlose erscheint unverletzt und auch nach professionieller
Untersuchung bei guter Gesundheit -- mit Ausnahme des Umstandes, daß
er Bewußtlos ist. Er wurde offenbar nicht ausgeraubt, denn jede Menge
einfach zu entwendender Wertgegenstände und Waffen trägt er noch bei
sich: Eine Geldbörse mit um die 100\,SS, Ohrringe, Gewandspange, ein
DaDao, ein Tanto\dots Die Tunika ist allerdings geöffnet, als ob
jemand etwas aus der Innentasche entfernt hätte. Im Nacken steckt ein
kleiner Blasrohrpfeil.  Die vermutliche Schußrichtung deutet auf ein
verlassenes Lagerhaus hin. Mit \emph{Spurenlesen} kann man
feststellen, daß sich dort wohl jemand auf die Lauer gelegt hat, aber
der Täter ist bereits über alle Berge, ohne weiter Spuren zu
hinterlassen und ohne von Zeugen gesehen worden zu sein.

Ein erfahrener Arzt (oder eine SpF mit \emph{Giftmischen}) kann
feststellen, daß der Pfeil mit einem sehr starken Narkosemittel
vergiftet ist, daß nur äußerst schwer zu beschaffen ist, dafür aber
auch außer Tiefschlaf keine ernsthaften Nebenwirkungen hat.

Nach zwei Stunden (eine Stunde nach erfolgreichem
\textbf{EW:Heilkunde} oder Betreuung durch einen Arzt) wacht der
Bewußtlose auf und stellt sich als \textbf{HaiFeng}, Kapitän im
Dienste der Familie Ming, vor. Bald erinnert er sich daran, daß er
\textbf{MingMing} eine Nachricht überbringen wollte, die ihm aber
offenbar entwendet worden ist.

Das Attentat wurde von einem Agenten der KuroScha mit großer
Professionalität ausgeführt. Die Nachricht befindet sich zum
Zeitpunkt, an dem die SpF HaiFeng finden, bereits im Hause der
\textbf{Xuang}. Sie wird dort zur Kenntnis genommen und vernichtet.

\subsection{Ärztliche Hilfe}

Wollen die SpF einen Arzt bemühen (um HaiFeng zu verarzten oder
Auskunft über das Gift zu erhalten), werden sie an die berühmte
YiScheng MuLanLu verwiesen (oder kennen sie vielleicht auch
schon). Sie hat ihre Praxis unweit des Seemarkten (Nr.\,9 auf dem Plan
[DSH\,42]). Die SpF müssen allerdings erfahren, daß MuLan vor vier
Tagen zu ihrem Anwesen am SchuSchan aufgebrochen ist -- mit vorerst
unbestimmten Rückkehrdatum. Ihre erfahrenen Assistenten können
allerdings die wohl recht einfachen Wünsche der SpF zur vollsten
Zufriedenheit zu günstigen Preisen erfüllen.

MuLan ist als Mitglied des \emph{Weißen Lotus} bzw. der \emph{Weißen
  Orchidee} zu Hilfe gerufen worden, um sich um die zahlreichen
Verwundeten zu kümmern, die die Orchideenklingen durch die jetzt in
die heiße Phase tretenden Aktionen gegen die Xuang zu beklagen
haben. Als wohlhabende Bürgerin KuenKungs verfügt sie über ein
Sommerhaus an den Hängen des SchuSchans, das mit seiner Nähe zur
Provinz Yü ideal gelegen ist als "`Krankenlager"'.

\subsection{Ming}

\subsection{Bei Gericht}

\subsection{Familie Xuang}

\subsection{Weitere Orte}

Mit den Informationen in [DSH] und [PdF] verfügt der SpL über
ausreichend Mittel, um auf weitere Ermittlungen der SpF in KuenKung
einzugehen. Allerdings gibt es für die SpF nicht viel mehr zu
gewinnen. Im Gegenteil, die Xuang könnten auf sie aufmerksam werden
und zu dem Schluß kommen, daß sie etwas "`Zuneigung"' brauchen. Was
das bedeutet, kann man sich anhand [DSH\,57f] ausmalen. Möglicherweise
schicken die Xuang den SpF auch einen Spion hinterher. Solange die SpF
aber unauffällig bleiben, fühlen die Xuang sich sicher und glauben,
daß keine weiteren Maßnahmen nötig seien.

\section{Der Weg nach OnchiRa}

Wahrscheinlich werden die SpF zunächst nach OnchiRa reisen, womit sich
dieser Abschnitt beschäftigt. Möglicherweise wollen sie aber auch
MuLanLu nachreisen oder den Aktivitäten der Xuang nachgehen. In diesem
Falle können sie über die gut ausgebauten Straßen zum SchuSchan
bzw. in Richtung ChüanHoSchang reisen. Der SpL kann die Reise dann
anhand der Informationen in Abschnitt~\ref{yue} gestalten.

\subsection{Zur See}

% Wako

% Meerjungfrauen

% Wetter!

\subsection{Zu Land}

\section{Die Schöne und das Biest}

\section{Zwischen SchuSchan und SongSchan}
\label{yue}

% TODO: Wetter!

\subsection{Orte auf der Karte}

\begin{enumerate}
\item Winterlager von AMei. Hier hat AMei, die Älteste der Yü, mit
  ihren weiteren Familienkreis und Gefolge ihr Winterlager aufgeschlagen.
\end{enumerate}

\section{RenSchenYo -- die Ginsengwelt}

% Oszillation: Schnell 2343 und jetzt, langsam 2385

\appendix

\section{AEP-Vergabe, Belohnungen und Lernmöglichkeiten}

\section{Mögliche Folgeabenteuer}

\section{NSpF}

\subsection{Ming}

\subsection{TsueChen}

% Direktor LangTsong im Finanzministerium HuPu im Range eines
% Silberfasans BoZjan (in DSH im Range eines Pfaus

\subsection{KokoRoDi und weitere Orchideenklingen}

% Dea ex machina
% maskiert

%\end{multicols}
\end{document}
