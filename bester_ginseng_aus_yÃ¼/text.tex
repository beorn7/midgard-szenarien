\documentclass[
a4paper,
twoside,
DIV=calc,
BCOR=4mm,
fontsize=9pt,
twocolumn=on,
titlepage=on,
parskip=half
]{scrartcl}
\usepackage{german,multicol,supertabular,graphicx,hyperref}
\usepackage[utf8]{inputenc}
%\usepackage[T1]{fontenc}
%\usepackage{helvet}
%\usepackage{mathphv}

%\renewcommand{\familydefault}{phv}
%\renewcommand{\familydefault}{cmss}
%\setlength{\columnsep}{7mm}

%\usepackage{anysize}
%          {left}{right}{top}{bottom}
%\marginsize{2.0cm}{2.0cm}{2.0cm}{2.0cm}
%\setlength{\footnotesep}{5mm}

\hyphenation{NSpF}
\hyphenation{Re-gel-mo-di-fi-ka-tio-nen}

\newcommand{\midgard}{\textsc{Midgard}}

\author{Bj"orn~Rabenstein\\\href{mailto:bjoern@rabenste.in}{\nolinkurl{bjoern@rabenste.in}}}
\title{Bester Ginseng aus Yü}
\subtitle{\midgard-Abenteuer für 3--5 SpF der Grade 1--3}

\publishers{\fbox{%
\parbox{\columnwidth}{%
Dieses Material steht unter der Creative-Commons-Lizenz \emph{Namensnennung
-- Nicht-kommerziell -- Weitergabe unter gleichen Bedingungen 4.0
International}. Um eine Kopie dieser Lizenz zu sehen, besuchen Sie
\url{http://creativecommons.org/licenses/by-nc-sa/4.0/}.
}%
}}


\begin{document}

\maketitle


\tableofcontents

%\begin{multicols}{2}

\section{Einleitung}

Im Januar 2015 erfüllte ich mir so eine Art Jugendtraum und begann,
eine KanThaiPan-Kampagne zu leiten, in deren Zentrum die "`vier
Klassiker"', also die vier Richter-Di-Abenteuer, stehen sollten. Ein
Vorteil der späten Traumerfüllung war, daß wir wirklich mit dem
\emph{Fall des Mondloses} anfangen konnten. Dadurch stellte sich aber
die Frage, wie wir die gut zehnjährige Lücke zwischen dem
\emph{Mondlos} und den \emph{Füchsen} überbrücken sollten. Selbst mit
einem bißchen Füllmaterial und Tricks blieb bei uns eine Lücke von
2392\,nL bis 2400\,nL. Schickte man die SpF wirklich so lange auf
Abenteuer aus, dann wären sie danach so ca. auf Gr9, was dann doch
etwas heftig wäre für den Rest der Kampagne, ganz abgesehen mal davon,
wie lange man dafür bräuchte. Natürlich hätte ich jetzt die SpF acht
Jahre lang ihrem "`normalen"' Leben nachgehen lassen können, oder ich
hätte die Geschichte umschreiben können und die anderen
Richter-Di-Abenteuer früher stattfinden lassen. Beide Möglichkeiten
fand ich aber aus verschiedenen Gründen unbefriedigend. So habe ich
aus der Not eine Tugend gemacht und dieses Abenteuer entworfen, wo als
Teil der Handlung die SpF einen Zeitsprung unternehmen werden, und das
sogar ohne die naheliegende \emph{Mach über die Zeit}, sondern unter
Ausnutzung bekannter Eigenschaften des Multiversums.

Um mit dieser Klappe gleich noch ein paar andere Fliegen zu schlagen,
passieren noch ein paar andere Dinge:

\begin{itemize}
\item Die SpF lernen Ming kennen, den wichtigen Protagonisten aus
  \emph{Kurai-Anat, das Schwarze Herz}. Das ermöglicht es, später den
  Kernteil dieses Abenteuers auch mit einer reinen KanThai-Gruppe zu
  spielen. Ein paar Szenen aus dem Abenteuer, die sonst gar nicht mehr
  zur Geltung kämen, können hier auch gleich verwurstet werden.
\item In den \emph{Perlen der Füchse} schmuggelt der \emph{Weiße
    Lotus} Gingseng in weitem Bogen von irgendwo am Schattenmeer durch
  Minangpahit über den Seeweg nach KuenKung. Dabei liegt doch der
  Hauptproduzent von Gingseng, die Provinz Yü, mehr oder weniger
  direkt vor den Toren der Stadt. Dieses Abenteuer liefert eine
  Erklärung, wie es dazu kommen konnte. Obendrein erfährt man ein
  wenig über das Verhältnis zwischen \emph{Weißer Orchidee} und
  \emph{Weißem Lotus}. Letzterer war ja zu Beginn des Abenteuers noch
  keine eigenständige Geheimgesellschaft, sondern lediglich eine Loge
  von ersterer.
\end{itemize}

Ich kann mir vorstellen, daß auch andere Gruppen in der gleichen
Situation stecken. Daher habe ich mich entschlossen, das Abenteuer
hiermit der Allgemeinheit zur Verfüngung zu stellen. Ich bitte, den
etwas rohen Ausarbeitungszustand zu verzeihen. Ich habe im
wesentlichen auf die Schnelle meine Notizen ausformuliert. Karten und
Skizzen sind in dem beklagenswerten Zustand eingescannt worden, in dem
sie mein bescheidenes manuelles Zeichentalent hinterlassen hat.

Verweise auf Regelbücher, das KanThaiPan-Quellenbuch und diverse
Abenteuerhefte erfolgen nach dem Schema [\emph{XXX n}], wobei
\emph{XXX} die Abürzung für das Buch bzw. Heft ist (siehe folgende
Liste) und \emph{n} die Seitenzahl im jeweiligen Werk.

\begin{description}
\item[DFR] \emph{Das Fantasy-Rollenspiel} -- \midgard-Regelwerk 4. Auflage.
\item[DSH] \emph{Kurai-Anat, das Schwarze Herz} -- Abenteuer.
\item[KTP] \emph{Unter dem Schirm des Jadekaisers} -- Quellenbuch
  KanThaiPan 2. Auflage.
\item[MdS] \emph{Meister der Sphären} -- \midgard-Regelwerk 4. Auflage.
\item[MSS] \emph{Mord am Schwarzdorn-See} -- Abenteuer.
\item[PdF] \emph{Die Perlen der Füchse} -- Abenteuer.
\end{description}

Dieses Abenteuer beziehte sich auf die 4. Auflage des \midgard-Regelwerkes.

\section{Überblick}

Das KuraiAnat sorgt sich um die Versorgung mit Ginseng. Daß
vergleichsweise unabhängige Händler aus den Küstenregionen, denen
sogar die Zusammenarbeit mit feindlichen Geheimgesellschaften wie der
\emph{Weißen Orchidee} zuzutrauen ist, wesentliche Teile des
Ginsenghandels unter ihrer Kontrolle bringen, ist eine nicht zu
unterschätzende Gefahr. Der schwarze Adept TsueChen wird beauftragt,
die Sache nachhaltig in Ordnung zu bringen. Dafür verbündet er sich
mit der dem KuraiAnat loyal ergebenen Familie Xuan aus KuenKung, die
bereits einen guten Teil des Ginsenghandels mit dem Volk der Yü, das
den meisten Ginseng in KanThaiPan produziert, kontrolliert. Zusammen
mit den Xuan übt er Druck auf die Yü aus, ihre Ginsengernte nur noch
an die Familie Xuan zu verkaufen. Mittelfristig soll dieses Monopol
durch einen kaiserlichen Erlaß Gesetzeskraft erlangen.

Die Yü sind nicht zentral organisiert, aber sie haben eine Älteste,
deren Rat von den Yü nicht leichtfertig in den Wind geschlagen
wird. Die aktuelle Amtsinhaberin ist AMei. Auf sie üben TsueChen und
LiXuan, der zuständige Händler der Familie Xuan, besonders viel
Druck aus. Ihr Plan scheint aufzugehen, den AMei zeigt sich einsichtig
und rät ihren Volksgenossen, den "`Vorschlag"' der Xuan anzunehmen.

Die \emph{Weiße Orchidee}, die sich in der Tat im Ginsenghandel
engagiert, ist davon gar nicht begeistert. Sie versuchen einerseits,
Ginsengsammler der Yü auf ihre Seite zu bringen. Andererseits
sabotieren sie den Ginsenghandel der Xuan durch Überfälle auf ihre
Händler. Die \emph{Weiße Orchidee} wird insgeheim von der
Fürstenfamilie von KuenKung, den Tchung, unterstützt, ebenso von ihrer
vorwiegend durch Beamte und etablierte Personen des öffentlichen
Lebens gebildeten Loge des \emph{Weißen Lotus}. Insbesondere beim
\emph{Weißen Lotus} regen sich aber immer mehr Zweifel am Erfolg
dieser Strategie.

DiFang ist ein besonders erfolgreicher Ginsengsammler, der
traditionell mit der Familie Ming aus KuenKung zusammenarbeitet. Kurz
vor Beginn der Handlung des Abenteuers hat er sich aber entschlossen,
dem Drängen AMeis nachzugeben und seinen Ginseng nur noch an die Xuan
zu liefern. Er schreibt eine Nachricht nieder, die die Hintergründe
seiner Entscheidung sehr klar beschreibt, und gibt sie HaiFeng, dem
Kapitän im Diensten der Familie Ming, mit. Die Xuan bekommen davon
Wind. Weil sie nicht wollen, daß Beweise für ihre Aktivitäten an das
Licht der Öffentlichkeit geraten, überfallen sie HaiFeng und nehmen
ihm die versiegelte Botschaft ab.

An dieser Stelle betreten die SpF die Bühne des Geschehens. Sie finden
den bewußtlosen HaiFeng und werden früher oder später vom Händler
MingMing, dem Oberhaupt der Familie Ming, beauftragt, den Verbleib von
DiFang aufzuklären.

Die Ermittlungen führen die SpF in die Provinz Yü, und zwar
wahrscheinlich zunächst in das Fischerdorf OnchiRa, wo sie die Chance
erhalten, Zeuge des Wirkens eines YamaOni zu werden, der die Frau des
Dorfvorstehers zu seiner Geliebten erkoren hat.

DiFang hat ein Geheimnis: Er findet die großen Mengen besonders
qualitativen Ginsengs nicht im Land der Yü, ja noch nicht mal auf
Midgard. Er hat Kenntnis von einem uralten permanenten Weltentor nach
RenSchenYo, einer Urwelt, von der der Ginseng ursprünglich stammt. Den
Bedrängungen sowohl durch die Xuan als auch durch die \emph{Weiße
  Orchidee} entzieht er sich durch Flucht nach RenSchenYo. Dort wird
er Opfer eines Streiches des (weiblichen) Dschinns BaiSseYuen, ihres
Zeichens Lehrmeisterin von Ming, der insgeheim ein Elementarbeschwörer
ist. Er sitzt vorerst auf RenSchenYo fest. Da die Zeit in einer Urwelt
schneller vergeht als auf Midgard, haben die SpF auf Midgard viel
Zeit, die Spur DiFangs aufzunehmen oder auf andere Art und Weise auf
das Weltentor aufmerksam zu werden und schließlich DiFang aus seiner
mißlichen Lage zu retten.

Bis es dazu kommt, können die SpF noch so einiges über die politischen
Ereignisse rund um den Ginsenghandel herausfinden. Auch könnten sie
KokoRo über den Weg laufen, der vermißten Schwester einer der SpF, die
mittlerweile zu einer Orchideenklinge ausgebildet worden ist.

Während die SpF DiFang aus seiner Lage auf RenSchenYo retten, vergehen
auf Midgard knapp acht Jahre. Bei ihrer Rückkehr nach Midgard finden
sie einen komplett durch die Xuan monopolisierten Ginsenghandel durch
die Yü vor. Ming hat sich mittlerweile anderen Handelsaktivitäten,
insbesondere dem Fernhandel mit dem Westen Midgards,
zugewandt. Dennoch ist er hocherfreut über die Errettung seines
Freundes DiFang und sehr begierig, von den Abenteuern der SpF zu
erfahren. Der \emph{Weiße Lotus} hat sich weitgehend von der
\emph{Weißen Orchidee} gelöst und ähnelt nun eher einer eigenen
Geheimgesellschaft. Er hat ausländische Quellen für Ginseng
erschlossen, während die \emph{Weiße Orchidee} nur noch sehr
halbherzig versucht, direkt gegen den Ginsenghandel der Xuan
vorzugehen.

In vielen der folgenden Abschnitte finden sich Absätze, die mit
\textsc{Zukunft:} beginnen. Diese beschreiben Entwicklungen und
Ereignisse, die erst stattfinden, wenn die SpF sich auf RenSchenYo
befinden und etliche Jahre auf Midgard überspringen.  Alle zukünftigen
Ereignisse werden so beschrieben, wie sie ohne Eingreifen der SpF
stattfänden. Je nach Handlungen der SpF müssen Entwicklungen und
Ereignisse abgeändert werden. Das Abenteuer geht desweiteren davon
aus, daß die SpF im Winter 2400 wieder auf Midgard eintreffen. Der SpL
kann dies an die Bedürfnisse seiner Kampagne anpassen, muß dann aber
entsprechende Veränderungen vornehmen.

\section{Dramatis personae}

\begin{description}
\item[MingMing.] Händler der Weiße-Schlangen-Gilde in KuenKung und
  heimlicher Elementarbeschwörer.
\item[DiFang.] Ginsengsammler aus dem Volk der Yü. Geschäftspartner
  und Vertrauter Mings.
\item[HaiFeng.] Kapitän in Diensten Mings. Opfer der KuroScha.
\item[ChenXuan.] Wenig integrer Richter von KuenKung, Vorgänger von
  DiYung.
\item[LiXuan.] Händler aus KuenKung. Spezialist für den
  Ginsenghandel mit den Yü.
\item[AMei.] Älteste der Yü. Ansprechpartnerin von LiXuan.
\item[TsueChen.] Schwarzer Adept mit stetig sich verbesserndem
  Verhältnis zur Familie Xuan. Zuständig für die Ginsengversorgung
  des KuraiAnat.
\item[MuLanLu.] Renommierteste YiScheng (Ärztin) von
  KuenKung. Mitglied des \emph{Weißen Lotus}.
\item[KwanLi.] Misogyner Dorfältester von OnchiRa.
\item[MeiHua.] Seine Frau.
\item[Yao.] Ihr heimlicher Liebhaber. In Wirklichkeit ein YamaOtoko.
\item[KokoRo.] Orchideenklinge, Schwester einer SpF, vermeintlich
  den \emph{Weg der Tausend} gegangen.
\item[BaiSseYuen -- "`Weiße Wolke"'.] Dschinn. Lehrmeisterin von Ming.
\item[MiTo.] Priester des KuTuh, z.Zt. Abt im Kloster DaKuMi. Verwandt
  mit den Xuan in KuenKung, wo er im Laufe der 90er Jahre zum
  Hohepriester des örtlichen Tempel der Düsternis aufsteigen wird.
\end{description}

Agenten der \emph{Weißen Orchidee}, Mitglieder der Familie Xuan und
deren Schergen, KuroScha, Ginsengsammler der Yü, Meerjungfrauen, Wako,
Hunar, Luftrochen\dots

\section{Zeitablauf}

Die Datumsangaben erfolgen nach westlichem Muster (mit dem Bärenmond
als ersten Monat des Jahres). Zur Umrechnung in kanthanische
Jahresangaben siehe folgende Liste.

\begin{itemize}
\item 01. Wolf 2391\,nL -- 28. Draug 2392\,nL: GuiYou (Hahn)
\item 01. Wolf 2392\,nL -- 28. Draug 2393\,nL: JiaXu (Hund)
\item 01. Wolf 2393\,nL -- 28. Draug 2394\,nL: YiHai (Eber/Schwein)
\item 01. Wolf 2394\,nL -- 28. Draug 2395\,nL: BingZi (Ratte)
\item 01. Wolf 2395\,nL -- 28. Draug 2396\,nL: DingChou (Ochse)
\item 01. Wolf 2396\,nL -- 28. Draug 2397\,nL: WuYin (Tiger)
\item 01. Wolf 2397\,nL -- 28. Draug 2398\,nL: JiMao (Hase)
\item 01. Wolf 2398\,nL -- 28. Draug 2399\,nL: GengLong (Drache)
\item 01. Wolf 2399\,nL -- 28. Draug 2400\,nL: XinSi (Schlange)
\item 01. Wolf 2400\,nL -- 28. Draug 2401\,nL: RenWu (Pferd)
\end{itemize}

\subsection{Vor Beginn des Abenteuers}

\begin{description}
\item[In grauer Vorzeit:] Die Arracht identifizieren die
  Neun-Blumen-Berge als besonders geeignet, ein Tor in eine Urwelt der
  Luft/Holz-Sphäre zu errichten. Im Süden der heutigen Provinz Yü
  errichten sie schließlich ein Weltentor nach RenSchenYo. Die
  wichtigste Entdeckung auf dieser Welt war der Ginseng, der von den
  Arracht auch auf Midgard angepflanzt wurde, allerdings mit wesentlich
  geringerem Erfolg.
\item[um 1000\,vL:] Mit der Entwicklung menschlicher Zivilisation in
  KanThaiPan entsteht auch das Volk der Yü samt ihrer Religion des
  Ginsenggottes. Die Priester belegen das Weltentor mit einem Tabu und
  machen es für Menschen unauffindbar.
\item[2337\,nL:] Der junge Ziegenhirte DiFang entdeckt beim Verfolgen
  einer Ziege das Weltentor. Er folgt seiner Ziege nach RenSchenYo,
  kann aber das Tor nicht umpolen, um zurückzukehren.
\item[2343\,nL:] Durch einen Glücksfall kann DiFang einen Baumriesen
  beim Umpolen des Tores beobachten und so nach Midgard zurückkehren,
  wo in den wenigen Stunden, die er auf RenSchenYo verbracht hat,
  sechs Jahre vergangen sind. Fortan nutzt DiFang die Kenntnis über
  Ort und Umpolung des Weltentores, um zu einem legendären
  Ginsengsammler aufzusteigen. Zum Handelspartner seines Vertrauens
  entwickelt sich die Familie Ming aus KuenKung.
\item[2385\,nL:] Der junge MingMing widmet sich dem Studium des
  Multiversums. DiFang weiht ihn in sein Geheimnis ein und ermöglicht
  ihm einen Besuch RenSchenYos, wo er schließlich auf seine spätere
  übernatürliche Lehrerin BaiSseYuen trifft.
\item[2391\,nL:] Der Schwarze Adept TsueChen wird vom Kronrat zum
  Beauftragten für den Ginsenghandel ernannt. Seine Aufgabe ist es,
  die Ginsengversorgung des KuraiAnat sicherzustellen. Er nimmt mit
  der Familie Xuan auf, die bereits einen Großteil des Ginsenghandels
  mit den Yü kontrolliert.
\item[Herbst 2392\,nL:] Mit TsueChens Unterstützung übt LiXuan
  zunehmend Druck auf die Yü aus, nur noch mit der Familie Xuan zu
  handeln. Die \emph{Weiße Orchidee} beginnt, den Ginsenghandel der
  Familie Xuan aktiv zu sabotieren und Ginseng in eigene Kanäle
  umzulenken.
\item[1. Draug 2392\,nL:] DiFang, der sich mittlerweile außerstande
  sieht, gegen den Druck der anderen Yü weiterhin Ginseng an Ming zu
  liefern, trift in OnchiRa ein, um dem dort wartenden HaiFeng statt
  der erwarteten Ginsenglieferung ein versiegeltes Schreiben zu
  überreichen. Kurz nach Abfahrt HaiFengs treffen drei Xuan-SaMurai
  ein, die die Aufgabe haben, DiFangs Geheimnis zu enträtseln. Sie
  bekommen mit, daß DiFang HaiFeng einen Brief übergeben hat. Diese
  Information übermitteln sie per Brieftaube nach KuenKung.
\item[2. Draug 2392\,nL:] DiFang verläßt OnchiRa, heimlich verfolgt
  von den drei Xuan-SaMurai.
\item[3. Draug 2392\,nL:] DiFang trifft im Winterlager AMeis ein und
  verhandelt mit ihr über die zukünftigen Konditionen.
\item[4. Draug 2392\,nL:] DiFang reist weiter in Richtung Schattental,
  weiterhin von den SaMurai verfolgt.  Eine Patrouille von vier
  Orchideenklingen (unter ihnen KokoRo), die ebenfalls auf der Suche
  nach DiFang sind und seine Spur bei AMeis Lager aufnehmen wollten,
  heftet sich unauffällig an die Fersen der SaMurai. -- Gegen Abend
  kommt HaiFeng in KuenKung an und will sofort den Brief an Ming
  übergeben. Noch im Hafenbereich betäubt ihn aber hinterrücks ein
  KuroScha im Auftrag der Xuans und entwendet den Brief. Die SpF
  finden den betäubten HaiFeng.
\end{description}

\subsection{Nach Beginn des Abenteuers}

Dies ist der wahrscheinliche Ablauf. Je nach Aktionen der SpF kann
sich natürlich einiges ändern.

\begin{description}
\item[5. Draug 2392\,nL:] Ming zeigt den Überfall auf HaiFeng bei
  Gericht an, wobei ihm klar wird, daß von Seiten der Behörden keine
  Bemühungen um Aufklärung zu erwarten sind. Spätestens jetzt
  beauftragt er die SpF, dem Fall nachzugehen. -- Auf RenSchenYo wird
  DiFang von BaiSseYuen auf die nächste Insel entführt. Da das Tor
  immer noch umgepolt ist, dringen ab jetzt gelegentlich Luftrochen
  nach Midgard vor.
\item[6. Draug 2392\,nL:] Am Eingang des Schattentals bemerkt DiFang
  seine Verfolger, die ihn daraufhin gefangennehmen wollen. Dies ist
  für die Orchideenklingen Anlaß, nun einzugreifen. Sie töten zwei der
  SaMurai, der dritte entkommt Richtung KuTuh-Kloster. Die
  Orchideenklingen bedrängen dann DiFang, sich ihrer Seite
  anzuschließen. DiFang gerät in Panik und flieht durchs Weltentor,
  was er daraufhin sofort umpolt, um seine Verfolger
  abzuschneiden. Die Orchideenklingen schlagen ein verborgenes Lager
  auf, um die weitere Entwicklung zu beobachten. DiFangs Kummer lockt
  BaiSseYuen an, die sich gerade auf RenSchenYo aufhält und ihn
  durch einen Streich vorläufig dort festsetzt.
\item[7. Draug 2392\,nL:] Frühestmögliches Eintreffen der SpF in
  OnchiRa. Mögliche Beobachtung des Zusammentreffens von Yao und
  MeiHua. (Auch an späteren Tagen möglich.)
\item[12. Draug 2392\,nL:] Frühestmögliches Eintreffen der SpF am
  Weltentor. Mögliches Eintreffen der Verstärkung der Xuan-Krieger am
  Weltentor.
\item[Drachenmond 2393\,nL:] MeiHua bringt in OnchiRa ihren Sohn
  Chang zur Welt.
\item[2394\,nL:] Es ergeht eine kaiserliche Verordnung zur
  Monopolisierung des Ginsenghandels mit den Yü. Die dahingehende
  Lizenz wird der Familie Xuan erteilt. Die Yü beugen sich, und der
  \emph{Weiße Lotus} erschließt andere Quellen für den Ginsenghandel,
  während die \emph{Weiße Orchidee} noch längere Zeit versucht, das
  Monopol durch Überfälle und Verbündete bei den Yü zu unterlaufen.
\item[3497\,nL:] Richter ChenXuan wir pensioniert. DiYung wird (nicht
  ganz ohne Einflußnahme seiner Familie) zum neuen Richter von
  KuenKung ernannt.
\item[Trollmond 2400\,nL:] Etwa zu diesem Zeitpunkt kehren die SpF
  und DiFang nach Midgard zurück. -- In OnchiRa zeigt Chang immer mehr
  bestialische Züge. Es werd Zeit für Yao, seinen Sohn an sich zu
  nehmen\dots
\end{description}

\section{Ermittlungen in KuenKung}

Sowohl [DSH] wie auch [PdF] enthalten Beschreibungen der Stadt
KuenKung, die der SpL hier verwenden kann.

\subsection{Handlungsunfähig im Handelshafen}

Am Abend des 4. Draug 2392\,nL (also am 4. Tag des 13. Monats im Jahr
GuiYou) finden die SpF im Handelshafenviertel einen gut gekleideten
Seemann bewußtlos in einer dunklen Ecke liegend (z.\,B. auf dem
Heimweg, wenn sie in der \emph{Meeresbrise} [PdF\,15f] abgestiegen
sind).

Der Bewußtlose erscheint unverletzt und auch nach professionieller
Untersuchung bei guter Gesundheit -- mit Ausnahme des Umstandes, daß
er Bewußtlos ist. Er wurde offenbar nicht ausgeraubt, denn jede Menge
einfach zu entwendender Wertgegenstände und Waffen trägt er noch bei
sich: Eine Geldbörse mit um die 100\,SS, Ohrringe, Gewandspange, ein
DaDao, ein Tanto\dots Die Tunika ist allerdings geöffnet, als ob
jemand etwas aus der Innentasche entfernt hätte. Im Nacken steckt ein
kleiner Blasrohrpfeil.  Die vermutliche Schußrichtung deutet auf ein
verlassenes Lagerhaus hin. Mit \emph{Spurenlesen} kann man
feststellen, daß sich dort wohl jemand auf die Lauer gelegt hat, aber
der Täter ist bereits über alle Berge, ohne weiter Spuren zu
hinterlassen und ohne von Zeugen gesehen worden zu sein.

Ein erfahrener Arzt (oder eine SpF mit \emph{Giftmischen}) kann
feststellen, daß der Pfeil mit einem sehr starken Narkosemittel
vergiftet ist, daß nur äußerst schwer zu beschaffen ist, dafür aber
auch außer Tiefschlaf keine ernsthaften Nebenwirkungen hat.

Nach zwei Stunden (eine Stunde nach erfolgreichem
\textbf{EW:Heilkunde} oder Betreuung durch einen Arzt) wacht der
Bewußtlose auf und stellt sich als \textbf{HaiFeng}, Kapitän im
Dienste der Familie Ming, vor. Bald erinnert er sich daran, daß er
\textbf{MingMing} eine Nachricht überbringen wollte, die ihm aber
offenbar entwendet worden ist. Zunächst ist er noch etwas schwach auf
den Beinen und wäre über Begleitung zu Mings Haus sehr erfreut.

Das Attentat wurde von einem Agenten der KuroScha mit großer
Professionalität ausgeführt. Die Nachricht befindet sich zum
Zeitpunkt, an dem die SpF HaiFeng finden, bereits im Hause der
\textbf{Xuan}. Sie wird dort zur Kenntnis genommen und vernichtet.

\subsection{Ärztliche Hilfe}

Wollen die SpF einen Arzt bemühen (um z.\,B. HaiFeng behandeln zu
lassen oder Auskunft über das Gift zu erhalten), werden sie an die
berühmte YiScheng MuLanLu verwiesen (oder kennen sie vielleicht auch
schon). Sie hat ihre Praxis unweit des Seemarkten (Nr.\,9 auf dem Plan
[DSH\,42]). Die SpF müssen allerdings erfahren, daß MuLan vor vier
Tagen zu ihrem Anwesen am SchuSchan aufgebrochen ist -- mit vorerst
unbestimmten Rückkehrdatum. Ihre erfahrenen Assistenten können
allerdings die wohl recht einfachen Wünsche der SpF zur vollsten
Zufriedenheit und zu günstigen Preisen erfüllen.

MuLan ist als Mitglied des \emph{Weißen Lotus} bzw. der \emph{Weißen
  Orchidee} zu Hilfe gerufen worden, um sich um die zahlreichen
Verwundeten zu kümmern, die die Orchideenklingen durch die jetzt in
die heiße Phase tretenden Aktionen gegen die Xuan zu beklagen
haben. Als wohlhabende Bürgerin KuenKungs verfügt sie über ein
Sommerhaus an den Hängen des SchuSchans, das mit seiner Nähe zur
Provinz Yü ideal gelegen ist als "`Krankenlager"'. Siehe auch
Abschnitt~\ref{mulanlus-anwesen}.

\subsection{Ming}

Bringen die SpF HaiFeng selbst zu Mings Haus [DSH\,45ff], zeigt er
sich sehr besorgt und gleichzeitig dankbar für die Fürsorge der
SpF. Er erwartete eigentlich eine Ginsenglieferung. HaiFeng kann ihm
zwar berichten, daß DiFang statt Ginseng nur eine Nachricht übergeben
hat, hat den Inhalt des versiegelten Briefes aber natürlich nicht
eingesehen. DiFang wollte ihm ganz offenbar keine detaillierteren
Auskünfte geben.

Von sich aus stellt Ming keine Mutmaßungen über die Hintergründe
an. Fragen man ihn direkt nach den Xuan oder nach Konkurrenten im
Ginsenghandel, dann erwähnt er aber sofort, daß die Xuan den Großteil
des Handels mit den Yü kontrollierten. Sie operierten dabei über den
Landweg, während er mit DiFang, einem langjährigen Freund der Familie,
mit dem bereits sein Vater zusammengearbeitet habe, eine
Handelsverbindung über See unterhalte. DiFang sei zwar nur ein
einzelner Ginsengsammler, verschaffe ihm aber regelmäßig recht
beachtliche Mengen von Ginseng höchster Qualität, den er dann an
anspruchsvolle Kunden im ganzen Land weiterverkaufe. Die Übergabe
finde im Fischerdorf OnchiRa statt.

Ming ist in DiFangs Geheimnis eingeweiht, wird aber davon nichts
verraten. Dies hat er DiFang hoch und heilig versprochen. Außerdem ist
sein eigenes Geheimnis um seine Tätigkeit als Beschwörer mit DiFangs
Geheimnis direkt verknüpft.

Am nächsten Morgen wird Ming den Überfall bei Gericht anzeigen (siehe
nächster Abschnitt). Spätestens zu dem Zeitpunkt, wo ihm klar wird,
daß vonseiten des Gerichts keinerlei Untersuchungen zu erwarten sind,
wird er die SpF beauftragen, Licht in den Fall zu bringen. Diese
Beauftragung kann aber auch schon früher passieren, wenn die SpF
Interesse zeigen. Die Bezahlung muß ausgehandelt werden. Ming ist in
der Lage, Gefallen und Dienstleistungen zu vermitteln (inklusive
einiger Lehrmeister), würde sich aber auch auf eine erfolgsabhängige
Bezahlung wie z.\,B. einige Prozent Gewinnanteil am wieder in Gang
gebrachtem Ginsenghandel des nächsten Jahres. Sind die SpF eher auf
eine monetäre Belohnung aus, läßt sich auch das einrichten.  Grobe
Richtlinie: 1000 bis 2000\,GS pro Person im Erfolgsfalle, zuzüglich
Spesen, bei vorhandener Vertrauensbasis möglicherweise auch ein
Vorschuß.

Ming muß in wenigen Tagen zu einer Handelsreise in den Norden
aufbrechen und kann daher nicht selber bei Nachforschungen
mitwirken. Er möchte die SpF für den Fall vorbereiten, daß sie das
Weltentor benutzen müssen, ohne ihnen aber das Geheimnis seiner
Existenz zu verraten, bevor sie es selbst entdecken. Daher gibt er
ihnen einen kryptischen Hinweis, sobald er selber aufbricht oder die
SpF in die Provinz Yü abreisen:

\fbox{%
\parbox{\linewidth}{%
Solltet ihr auf eurer Reise vor verschlossenem Tor stehen, so sprecht
die Worte "`Kra'anesch"', und euch wird geöffnet werden.
}%
}

Mit einem \textbf{EW:Dunkle Sprache} kommt man zu dem Schluß, daß es
sich wohl um eine Wort aus der Dunklen Sprache handeln könnte, kann
die Bedeutung aber nicht ganz entschlüsseln. Irgendwas mit Luft und
Leben vielleicht?

\subsection{Bei Gericht}

Eine Beschreibung des Gerichts von KuenKung (für das Jahr 2402) findet
sich in [DSH\,20f]. Bei den Bediensteten muß man den Zeitunterschied
berücksichtigen. Ansonsten hat sich in den zehn Jahren zwischen 2392
und 2402 nicht viel verändert. Aktueller Richter ist ChenXuan, ein
Mitglied der Familie Xuan, der durchaus die Interessen seiner Familie
in seine Entscheidungen einfließen läßt. Er ist zwar nicht konkret in
das Vorgehen gegen Ming bzw. HaiFeng eingeweiht, kann sich aber genug
denken, so daß er den Fall erstmal auf die lange Bank schiebt und
später --~nach Rücksprache mit LiXuan~-- ganz zu den Akten legt. Es
wirkt auch halbegs glaubwürdig, daß das Gericht Wichtigeres zu tun
hat, als einen Fall zu untersuchen, bei dem lediglich ein Brief
entwendet wurde und ganz offensichtlich keinerlei andauernde
Verletzungen oder Verlust an Wertgegenständen zu beklagen ist.

ChenXuan bzw. seine Mitarbeiter wollen sich mit den SpF gar nicht erst
abgeben und geben keinerlei Hilfestellung bei möglichen
Recherchen. Schon gar nicht erlauben sie Zugang zum
Gerichtsarchiv. Das wäre nur offiziell vom Gericht beauftragten
XuSchau gestattet.

\subsection{Familie Xuan}

Es ist leicht in Erfahrung zu bringen, daß die Xuan eine
alteingesessene mächtige Familie ist, die seit Generationen in
Konkurrenz mit der aktuellen Herrscherfamilie Tchung steht. Der Handel
ist nur ein Teil ihrer Aktivitäten, und dabei engagieren sie sich
nicht in der größten ortsansässigen Gilde, der
\emph{Weißen-Schlangen-Gilde}, sondern sympathisieren eher mit der
\emph{Gilde des endlosen Knotens}, die ihre Hauptsitze in KueiChan
und YenChan hat.

Es bedarf genauerer Recherchen an geeigneten Orten und mittels
geeigneter Fertigkeiten wie z.\,B. \emph{Gassenwissen},
\emph{Geschäftstüchtigkeit}, \emph{Landeskunde}\dots, um die folgenden
Informationen herauszufinden (in aufsteigender Schwierigkeit):

\begin{itemize}
\item Die Familie hängt dem KuTuh-Glauben an und hat beste
  Verbindungen zum Tempel der Düsternis.
\item Während die Tchung eher für die Unabhängigkeit KuenKungs von der
  Zentralmacht eintreten und ganz bewußt vermeiden, einen Schwarzen
  Adepten als Berater zu haben, streben die Xuan eine engere Bindung
  an das KuraiAnat an.
\item MiTo ist eine Mitglied der Familie und ein talentierter
  KuTuh-Priester, der schon in recht jungen Jahren zum Abt des
  DaKuMi-Klosters in den Neun-Blumen-Bergen befördert wurde und nun
  auf die Position des Hohepriesters in KuenKung zuarbeitet.
\item Der Schwarze Adept TsueChen pflegt in letzter Zeit engere
  Kontakte zur Familie.
\end{itemize}

Suchen die SpF das Anwesen der Familie Xuan direkt auf, werden sie vom
Personal höflich aber bestimmt abgewiesen. Allzu hartnäckiges
Nachfragen erregt schließlich die Aufmerksamkeit der Familie und hat
die im nächsten Abschnitt beschriebenen Folgen.

\subsection{Weitere Orte}

Mit den Informationen in [DSH] und [PdF] verfügt der SpL über
ausreichend Mittel, um auf weitere Ermittlungen der SpF in KuenKung
einzugehen. Allerdings gibt es für die SpF nicht viel mehr zu
gewinnen. Im Gegenteil, die Xuan könnten auf sie aufmerksam werden
und zu dem Schluß kommen, daß sie etwas "`Zuneigung"' brauchen. Was
das bedeutet, kann man sich anhand [DSH\,57f] ausmalen. Möglicherweise
schicken die Xuan den SpF auch einen Spion hinterher. Solange die SpF
aber unauffällig bleiben, fühlen die Xuan sich sicher und glauben,
daß keine weiteren Maßnahmen nötig seien.

\section{Der Weg nach OnchiRa}

Wahrscheinlich werden die SpF zunächst nach OnchiRa reisen, womit sich
dieser Abschnitt beschäftigt. Möglicherweise wollen sie aber auch
MuLanLu nachreisen oder den Aktivitäten der Xuan nachgehen. In diesem
Falle können sie über die gut ausgebauten Straßen zum SchuSchan
bzw. in Richtung ChüanHoSchang reisen. Der SpL kann die Reise dann
anhand der Informationen in Abschnitt~\ref{yue} gestalten.

\subsection{Zur See}

Wenn der SpL es den SpF nicht zu schwer machen will, stellt Ming
HaiFeng samt Mannschaft und Schiff (die \emph{HoHsien} für die Reise
zur Verfügung. Die Crew ist erfahren und wehrhaft genug, so daß es
nicht zu Problemen mit den Wako (den kanthanischen Piraten) kommen
sollte. Will der SpL der Gruppe einen Kampf liefern, kann er einen
Piratenüberfall im Stile von [DSH\,30f] inszenieren. Besteht die
Gruppe aus erfahrenen Seeleuten, kann der SpL für HaiFeng auch andere
wichtige Pläne vorsehen, so daß die SpF die Seereise selber
organisieren müssen.

Das Wetter ist gutmütig, milde Temperaturen, kein Niederschlag und
mittelstarke, gut zum Segeln geeignete Winde. Die HoHsien erreicht
OnchiRa in drei Tagen. SpF brauchen vielleicht je nach Fähigkeiten
etwas länger.

In jedem Falle bietet es sich an, die Begegnung mit den Seejungfrauen
aus [DSH\,31ff] einzubauen. Der \emph{Rat}, den die Seejungfrauen
geben, muß natürlich abgeändert werden. Der SpL kann einen oder mehrere
der folgenden wählen, oder er denkt sich etwas Passendes für seine
Kampagne aus.

\begin{quote}
  Für eine kurze Weile reitet ihr\\
  auf einer kleinen hölzernen Wolke.\\
  Bald wird es eine große sein\\
  für eine lange Zeit.
\end{quote}

Das Schiff, die HoHsien, ist nach einer der acht Unsterblichen benannt
(was jeder, der über \emph{Kenntnis der fünf Klassiker} verfügt,
erkennt, sobald er über den Namen nachdenkt. Das Attribut dieser
Unsterblichen ist die Wolke. Die kleine hölzerne Wolke ist also die
HoHsien. Die großen hölzernen Wolken sind die schwebenden Inseln auf
RenSchenYo.

\begin{quote}
  Wenn die Frau eine Frau ist,\\
  der Mann aber kein Mann,\\
  ist es dennoch Liebe?
\end{quote}

Dies bezieht sich auf MeiHua und Yao.

\begin{quote}
  Den Einsamen\\
  beschert die schwarze Nacht\\
  das stillste Wasser.
\end{quote}

Die Yü werden ihren Frieden nur finden, wenn sie mit den Schwarzen
Adepten kooperieren.

\subsection{Zu Land}

Zu Land drohen zwar kein Wako, aber die sogenannte Küstenstraße ist
nur ein besserer Trampelpfad. Jetzt im Winter ist er zwar zumeist
recht gut zu bereisen, durch das Marschland südlich von KoSchiRa führt
er aber auch um diese Jahreszeit über halb im Sumpf versunkene alte
Knüppelpfade. Siehe Abschnitt \ref{yue} für Informationen, wie der SpL
diese lange und mühselige Reise mit allerlei Annehmlichkeiten
ausfüllen kann.

\section{Die Schöne und das Biest}

\section{Zwischen SchuSchan und SongSchan}
\label{yue}

Es sind zwei Karten beigelegt. Die eine zeigt das, was die SpF leicht
in Erfahrung bringen können. Ming könnte ihnen z.\,B. so eine Karte
zur Verfügung stellen. Die andere zeigt alle Orte an, die für das
Abenteuer relevant sind. Selbst diese Karte ist nicht vollständig. Es
gibt wesentlich mehr in der Provinz zu entdecken, was in keiner der
beiden Karten eingezeichnet ist.

% TODO: Wetter!

Im folgenden werden besondere Orte von Nord nach Süd beschrieben,
gefolgt von einer Liste möglicher Begegnungen.

\subsection{See von Harima}

Soll das Abenteuer \emph{Das Schlemmermahl} [MSS\,5ff] nach KuroKegaTi
verlegt werden, bietet es sich an, den See von Harima hier an die
Hänge des SchuSchan zu versetzen.

\subsection{MuLanLus Anwesen}
\label{mulanlus-anwesen}

\subsection{Basislager der Orchideenklingen}

\subsection{Kloster DaKuMi}

\subsection{KoSchiRa}

\subsection{Das Marschland}

\subsection{Schattental und Weltentor}

% Schlüsselwort "`Kra'anesch"' ist der Name der Welt, wurde zu
% RenSchen, Ginseng, in KanThaiTun.

\subsection{OnchiRa}

Siehe Abschnitt \ref{biest} und [DSH\,34ff].

\subsection{AMeis Winterlager}

AMei, die Älteste der Yü, hat weit im Süden, unweit des SongSchan, mit
ihrem weiteren Familienkreis und Gefolge ihr Winterlager
aufgeschlagen.

\subsection{PadKu-Tempel auf dem SongSchan}

% Tor nach KueiLi.

\section{RenSchenYo -- die Ginsengwelt}

% Oszillation: Schnell 2343 und jetzt, langsam 2385

\appendix

\section{AEP-Vergabe, Belohnungen und Lernmöglichkeiten}

\section{Mögliche Folgeabenteuer}

\section{NSpF}

\subsection{Familie Ming und ihre Verbündeten}

% MingMing

% HaiFeng

% Seeleute

% BaiSseYuen

\subsection{Familie Xuan und ihre Verbündeten}

% LiXuan

% ChenXuan

% MiTo

% TsueChen

% Direktor LangTsong im Finanzministerium HuPu im Range eines
% Silberfasans BoZjan (in DSH im Range eines Pfaus

% SaMurai

\subsection{Yü}

Die Yü sind im allgemeinen etwas kleiner und zierlicher als der
durchschnittliche KanThai und haben eine deutlich dunklere
Hautfarbe. Dank des allgegenwärtigen Ginseng ist ihre Lebenserwartung
weitaus höher als üblich. Viele Yü werden über 100 Jahre alt.

% DiFang

% AMei

\subsection{In OnchiRa}

% KwanLi

% MeiHuaLi

% Yao

\subsection{HaLan -- die Orchideenklingen}

% KokoRoDi und weitere Orchideenklingen

% Dea ex machina
% maskiert

\subsection{Bewohner RenSchenYos}

% Luftrochen

% Hunar

% Arracht

%\end{multicols}
\end{document}
