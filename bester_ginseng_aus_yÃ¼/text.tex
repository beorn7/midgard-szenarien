\documentclass[12pt,a4paper,germanpar]{scrartcl}
\usepackage{german,anysize,multicol,supertabular,graphicx}
\usepackage[utf8]{inputenc}
%\usepackage[T1]{fontenc}
%\usepackage{helvet}
%\usepackage{mathphv}

%\renewcommand{\familydefault}{phv}
%\renewcommand{\familydefault}{cmss}
%\setlength{\columnsep}{7mm}

%          {left}{right}{top}{bottom}
%\marginsize{2.0cm}{2.0cm}{2.0cm}{2.0cm}
%\setlength{\footnotesep}{5mm}

\hyphenation{NSpF}
\hyphenation{Re-gel-mo-di-fi-ka-tio-nen}

\newcommand{\midgard}{\textsc{Midgard}}

\author{Bj"orn~Rabenstein\\bjoern@rabenste.in}
\title{Bester Ginseng aus Yü}
\subtitle{\midgard-Szenario für 3--5 SpF der Grade 1--3}

\begin{document}

\maketitle

\tableofcontents

%\begin{multicols}{2}

\section{Einleitung}


Verweise auf Regelbücher, das KanThaiPan-Quellenbuch und diverse
Szenarienhefte erfolgen nach dem Schema [\emph{XXX n}], wobei
\emph{XXX} die Abürzung für das Buch bzw. Heft ist (siehe folgende
Liste) und \emph{n} die Seitenzahl im jeweiligen Werk.

\begin{description}
\item[DFR] \emph{Das Fantasy-Rollenspiel} -- \midgard-Regelwerk 4. Auflage.
\item[DSH] \emph{Kurai-Anat, das Schwarze Herz} -- Abenteuer.
\item[KTP] \emph{Unter dem Schirm des Jadekaisers} -- Quellenbuch
  KanThaiPan 2. Auflage.
\item[MdS] \emph{Meister der Sphären} -- \midgard-Regelwerk 4. Auflage.
\item[PdF] \emph{Die Perlen der Füchse} -- Abenteuer.
\end{description}

Dieses Szenario beziehte sich auf die 4. Auflage des \midgard-Regelwerkes.

\section{Überblick}

\section{Dramatis personae}

\begin{description}
\item[MingMing.] Händler der Weiße-Schlangen-Gilde in KuengKung und
  heimlicher Elementarbeschwörer.
\item[DiFang.] Ginsengsammler aus dem Volk der Yü. Geschäftspartner
  und Vertrauter Mings.
\item[HaiFeng.] Kapitän in Diensten Mings. Opfer der KuroScha.
\item[ChenXuang.] Wenig integrer Richter von KuengKung, Vorgänger von
  DiYung.
\item[LiXuang.] Händler aus KuengKung. Spezialist für den
  Ginsenghandel mit den Yü.
\item[AMei.] Älteste der Yü. Ansprechpartnerin von LiXuan.
\item[TsueChen.] Schwarzer Adept mit stetig sich verbesserndem
  Verhältnis zur Familie Xuang. Zuständig für die Ginsengversorgung
  des KuraiAnat.
\item[KwanLi.] Misogyner Dorfältester von OnchiRa.
\item[MeiHua.] Seine Frau.
\item[Yao.] Ihr heimlicher Liebhaber. In Wirklichkeit ein YamaOtoko.
\item[KokoRoDi.] Orchideenklinge, Schwester einer SpF, vermeintlich
  den \emph{Weg der Tausend} gegangen.
\item[BaiSseYuen.] Dschinn. Lehrmeister von Ming.
\end{description}

Agenten der \emph{Weißen Orchidee}, Mitglieder der Familie Xuan und
deren Schergen, KuroScha, Ginsengsammler der Yü, Meerjungfrauen, Wako\dots

\section{Zeitablauf}

% 4.12.2392nL Beginn 4.13.GuiYou

\section{Ermittlungen in KuengKung}

\section{Der Weg nach OnchiRa}

% Wako

% Meerjungfrauen

\section{Tête-à-tête in OnchiRa}

\section{Zwischen SchuSchan und SongSchan}

\section{RenSchenYo -- die Ginsengwelt}

\appendix

\section{NSpF}

\subsection{Ming}

\subsection{KokoRoDi}

%\end{multicols}
\end{document}
